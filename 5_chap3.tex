\chapter*{Заключение}
\addcontentsline{toc}{chapter}{Заключение}

В результате выполнения работы было разработано Android-приложение для получения и сохранения геолокационных данных с использованием современных инструментов и технологий. Я опирался на репозиторий преподавателя \cite{GitHub}, на видеоролики из ютуба \cite{YouTube} и сайт Metonit \cite{Metanit}. В качестве фреймворка для пользовательского интерфейса использован \textbf{Jetpack Compose}, что позволило создать гибкий и адаптивный интерфейс с декларативным подходом.

Основной функционал приложения включает:
\begin{itemize}
    \item Автоматический запрос разрешений на доступ к геолокации через \textbf{Runtime Permissions}.
    \item Отображение текущих координат (широты и долготы) в реальном времени.
    \item Сохранение данных в формате \textbf{JSON} в системную папку \texttt{Downloads} через \textbf{MediaStore API}.
    \item Интеграцию с \textbf{FusedLocationProviderClient} для точного определения местоположения.
\end{itemize}

При разработке были применены лучшие практики проектирования:
\begin{itemize}
    \item Асинхронная обработка операций с использованием коллбеков и \textbf{Listeners}.
    \item Реализация устойчивости к изменениям конфигурации устройства.
    \item Организация кода с учётом принципов расширяемости и поддерживаемости.
\end{itemize}

Полученный результат демонстрирует корректную работу всех компонентов: успешное получение координат, их визуализацию в интерфейсе, сохранение в файл с уникальным именем, а также обработку ошибок при отсутствии разрешений или отключённой геолокации. Пользовательский интерфейс обеспечивает интуитивное взаимодействие, а система уведомлений через \textbf{Toast} информирует о статусе операций.

Разработка позволила получить практические навыки:
\begin{itemize}
    \item Работы с геолокацией в Android, включая взаимодействие с \textbf{Google Play Services}.
    \item Создания адаптивных интерфейсов на базе \textbf{Jetpack Compose}.
    \item Управления файловой системой через \textbf{ContentResolver} и \textbf{MediaStore}.
\end{itemize}

Перспективы развития приложения включают интеграцию с картографическими сервисами (например, \textbf{Google Maps}), реализацию фонового отслеживания локации, а также добавление синхронизации данных с облачными хранилищами. Данная работа подтверждает возможность создания устойчивых и масштабируемых решений в области мобильной разработки на платформе Android.
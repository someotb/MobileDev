\documentclass[a4paper,14pt,oneside,openany]{memoir}

%%% Задаем поля, отступы и межстрочный интервал %%%

\usepackage[left=30mm, right=15mm, top=20mm, bottom=20mm]{geometry} % Пакет geometry с аргументами для определения полей
\pagestyle{plain} % Убираем стандарные для данного класса верхние колонтитулы с заголовком текущей главы, оставляем только номер страницы снизу по центру
\parindent=1.25cm % Абзацный отступ 1.25 см, приблизительно равно пяти знакам, как по ГОСТ
\usepackage{indentfirst} % Добавляем отступ к первому абзацу
%\linespread{1.3} % Межстрочный интервал (наиболее близко к вордовскому полуторному) - тут вместо этого используется команда OnehalfSpacing*

%%% Задаем языковые параметры и шрифт %%%

\usepackage[english, russian]{babel}                % Настройки для русского языка как основного в тексте
\babelfont[russian]{rm}{Times New Roman}                     % TMR в качестве базового roman-щрифта



%%% Задаем стиль заголовков и подзаголовков в тексте %%%

\setsecnumdepth{subsection} % Номера разделов считать до третьего уровня включительно, т.е. нумеруются только главы, секции, подсекции
\renewcommand*{\chapterheadstart}{} % Переопределяем команду, задающую отступ над заголовком, чтобы отступа не было
\renewcommand*{\printchaptername}{} % Переопределяем команду, печатающую слово "Глава", чтобы оно не печалось
%\renewcommand*{\printchapternum}{} % То же самое для номера главы - тут не надо, номер главы оставляем
\renewcommand*{\chapnumfont}{\normalfont\bfseries} % Меняем стиль шрифта для номера главы: нормальный размер, полужирный
\renewcommand*{\afterchapternum}{\hspace{1em}} % Меняем разделитель между номером главы и названием
\renewcommand*{\printchaptertitle}{\normalfont\bfseries\centering\MakeUppercase} % Меняем стиль написания для заголовка главы: нормальный размер, полужирный, центрированный, заглавными буквами
\setbeforesecskip{20pt} % Задаем отступ перед заголовком секции
\setaftersecskip{20pt} % Ставим такой же отступ после заголовка секции
\setsecheadstyle{\raggedright\normalfont\bfseries} % Меняем стиль написания для заголовка секции: выравнивание по правому краю без переносов, нормальный размер, полужирный
\setbeforesubsecskip{20pt} % Задаем отступ перед заголовком подсекции
\setaftersubsecskip{20pt} % Ставим такой же отступ после заголовка подсекции
\setsubsecheadstyle{\raggedright\normalfont\bfseries}  % Меняем стиль написания для заголовка подсекции: выравнивание по правому краю без переносов, нормальный размер, полужирный

%%% Задаем параметры оглавления %%%

\addto\captionsrussian{\renewcommand\contentsname{Содержание}} % Меняем слово "Оглавление" на "Содержание"
\setrmarg{2.55em plus1fil} % Запрещаем переносы слов в оглавлении
%\setlength{\cftbeforechapterskip}{0pt} % Эта команда убирает интервал между заголовками глав - тут не надо, так красивее смотрится
\renewcommand{\aftertoctitle}{\afterchaptertitle \vspace{-\cftbeforechapterskip}} % Делаем отступ между словом "Содержание" и первой строкой таким же, как у заголовков глав
%\renewcommand*{\chapternumberline}[1]{} % Делаем так, чтобы номер главы не печатался - тут не надо
\renewcommand*{\cftchapternumwidth}{1.5em} % Ставим подходящий по размеру разделитель между номером главы и самим заголовком
\renewcommand*{\cftchapterfont}{\normalfont\MakeUppercase} % Названия глав обычным шрифтом заглавными буквами
\renewcommand*{\cftchapterpagefont}{\normalfont} % Номера страниц обычным шрифтом
\renewcommand*{\cftchapterdotsep}{\cftdotsep} % Делаем точки до номера страницы после названий глав
\renewcommand*{\cftdotsep}{1} % Задаем расстояние между точками
\renewcommand*{\cftchapterleader}{\cftdotfill{\cftchapterdotsep}} % Делаем точки стандартной формы (по умолчанию они "жирные")
\maxtocdepth{subsection} % В оглавление попадают только разделы первыхтрех уровней: главы, секции и подсекции

%%% Выравнивание и переносы %%%

%% http://tex.stackexchange.com/questions/241343/what-is-the-meaning-of-fussy-sloppy-emergencystretch-tolerance-hbadness
%% http://www.latex-community.org/forum/viewtopic.php?p=70342#p70342
\tolerance 1414
\hbadness 1414
\emergencystretch 1.5em                             % В случае проблем регулировать в первую очередь
\hfuzz 0.3pt
\vfuzz \hfuzz
%\dbottom
%\sloppy                                            % Избавляемся от переполнений
\clubpenalty=10000                                  % Запрещаем разрыв страницы после первой строки абзаца
\widowpenalty=10000                                 % Запрещаем разрыв страницы после последней строки абзаца
\brokenpenalty=4991                                 % Ограничение на разрыв страницы, если строка заканчивается переносом

%%% Объясняем компилятору, какие буквы русского алфавита можно использовать в перечислениях (подрисунках и нумерованных списках) %%%
%%% По ГОСТ нельзя использовать буквы ё, з, й, о, ч, ь, ы, ъ %%%
%%% Здесь также переопределены заглавные буквы, хотя в принципе они в документе не используются %%%

\makeatletter
    \def\russian@Alph#1{\ifcase#1\or
       А\or Б\or В\or Г\or Д\or Е\or Ж\or
       И\or К\or Л\or М\or Н\or
       П\or Р\or С\or Т\or У\or Ф\or Х\or
       Ц\or Ш\or Щ\or Э\or Ю\or Я\else\xpg@ill@value{#1}{russian@Alph}\fi}
    \def\russian@alph#1{\ifcase#1\or
       а\or б\or в\or г\or д\or е\or ж\or
       и\or к\or л\or м\or н\or
       п\or р\or с\or т\or у\or ф\or х\or
       ц\or ш\or щ\or э\or ю\or я\else\xpg@ill@value{#1}{russian@alph}\fi}
\makeatother

%%% Задаем параметры оформления рисунков и таблиц %%%

\usepackage{graphicx, caption, subcaption} % Подгружаем пакеты для работы с графикой и настройки подписей
\graphicspath{{images/}} % Определяем папку с рисунками
\captionsetup[figure]{font=small, width=\textwidth, name=Рисунок, justification=centering} % Задаем параметры подписей к рисункам: маленький шрифт (в данном случае 12pt), ширина равна ширине текста, полнотекстовая надпись "Рисунок", выравнивание по центру
\captionsetup[subfigure]{font=small} % Индексы подрисунков а), б) и так далее тоже шрифтом 12pt (по умолчанию делает еще меньше)
\captionsetup[table]{singlelinecheck=false,font=small,width=\textwidth,justification=justified} % Задаем параметры подписей к таблицам: запрещаем переносы, маленький шрифт (в данном случае 12pt), ширина равна ширине текста, выравнивание по ширине
\captiondelim{ --- } % Разделителем между номером рисунка/таблицы и текстом в подписи является длинное тире
\setkeys{Gin}{width=\textwidth} % По умолчанию размер всех добавляемых рисунков будет подгоняться под ширину текста
\renewcommand{\thesubfigure}{\asbuk{subfigure}} % Нумерация подрисунков строчными буквами кириллицы
%\setlength{\abovecaptionskip}{0pt} % Отбивка над подписью - тут не меняем
%\setlength{\belowcaptionskip}{0pt} % Отбивка под подписью - тут не меняем
\usepackage[section]{placeins} % Объекты типа float (рисунки/таблицы) не вылезают за границы секциии, в которой они объявлены

%%% Задаем параметры ссылок и гиперссылок %%% 

\usepackage{hyperref}                               % Подгружаем нужный пакет
\hypersetup{
    colorlinks=true,                                % Все ссылки и гиперссылки цветные
    linktoc=all,                                    % В оглавлении ссылки подключатся для всех отображаемых уровней
    linktocpage=true,                               % Ссылка - только номер страницы, а не весь заголовок (так выглядит аккуратнее)
    linkcolor=red,                                  % Цвет ссылок и гиперссылок - красный
    citecolor=red                                   % Цвет цитировний - красный
}

%%% Настраиваем отображение списков %%%

\usepackage{enumitem}                               % Подгружаем пакет для гибкой настройки списков
\renewcommand*{\labelitemi}{\normalfont{--}}        % В ненумерованных списках для пунктов используем короткое тире
\makeatletter
    \AddEnumerateCounter{\asbuk}{\russian@alph}     % Объясняем пакету enumitem, как использовать asbuk
\makeatother
\renewcommand{\labelenumii}{\asbuk{enumii})}        % Кириллица для второго уровня нумерации
\renewcommand{\labelenumiii}{\arabic{enumiii})}     % Арабские цифры для третьего уровня нумерации
\setlist{noitemsep, leftmargin=*}                   % Убираем интервалы между пунками одного уровня в списке
\setlist[1]{labelindent=\parindent}                 % Отступ у пунктов списка равен абзацному отступу
\setlist[2]{leftmargin=\parindent}                  % Плюс еще один такой же отступ для следующего уровня
\setlist[3]{leftmargin=\parindent}                  % И еще один для третьего уровня

%%% Счетчики для нумерации объектов %%%

\counterwithout{figure}{chapter}                    % Сквозная нумерация рисунков по документу
\counterwithout{equation}{chapter}                  % Сквозная нумерация математических выражений по документу
\counterwithout{table}{chapter}                     % Сквозная нумерация таблиц по документу

%%% Реализация библиографии пакетами biblatex и biblatex-gost с использованием движка biber %%%

\usepackage{csquotes} % Пакет для оформления сложных блоков цитирования (biblatex рекомендует его подключать)
\usepackage[%
backend=biber,                                      % Движок
bibencoding=utf8,                                   % Кодировка bib-файла
sorting=none,                                       % Настройка сортировки списка литературы
style=gost-numeric,                                 % Стиль цитирования и библиографии по ГОСТ
language=auto,                                      % Язык для каждой библиографической записи задается отдельно
autolang=other,                                     % Поддержка многоязычной библиографии
sortcites=true,                                     % Если в квадратных скобках несколько ссылок, то отображаться будут отсортированно
movenames=false,                                    % Не перемещать имена, они всегда в начале библиографической записи
maxnames=5,                                         % Максимальное отображаемое число авторов
minnames=3,                                         % До скольки сокращать число авторов, если их больше максимума
doi=false,                                          % Не отображать ссылки на DOI
isbn=false,                                         % Не показывать ISBN, ISSN, ISRN
]{biblatex}[2016/09/17]
\DeclareDelimFormat{bibinitdelim}{}                 % Убираем пробел между инициалами (Иванов И.И. вместо Иванов И. И.)
\addbibresource{bibl.bib}                           % Определяем файл с библиографией

%%% Скрипт, который автоматически подбирает язык (и, следовательно, формат) для каждой библиографической записи %%%
%%% Если в названии работы есть кириллица - меняем значение поля langid на russian %%%
%%% Все оставшиеся пустые места в поле langid заменяем на english %%%

\DeclareSourcemap{
  \maps[datatype=bibtex]{
    \map{
        \step[fieldsource=title, match=\regexp{^\P{Cyrillic}*\p{Cyrillic}.*}, final]
        \step[fieldset=langid, fieldvalue={russian}]
    }
    \map{
        \step[fieldset=langid, fieldvalue={english}]
    }
  }
}

%%% Прочие пакеты для расширения функционала %%%

\usepackage{longtable,ltcaption}                    % Длинные таблицы
\usepackage{multirow,makecell}                      % Улучшенное форматирование таблиц
\usepackage{booktabs}                               % Еще один пакет для красивых таблиц
\usepackage{soulutf8}                               % Поддержка переносоустойчивых подчёркиваний и зачёркиваний
\usepackage{icomma}                                 % Запятая в десятичных дробях
\usepackage{hyphenat}                               % Для красивых переносов
\usepackage{textcomp}                               % Поддержка "сложных" печатных символов типа значков иены, копирайта и т.д.
\usepackage[version=4]{mhchem}                      % Красивые химические уравнения
\usepackage{amsmath}                                % Усовершенствование отображения математических выражений 
\usepackage{listings}
\usepackage{xcolor}

\lstdefinelanguage{kotlin}{
  morekeywords={
    val, var, fun, for, if, else, in, when, return, class, object, interface, 
    override, import, package, is, as, try, catch, finally, throw, typeof,
    while, do, this, super, null, true, false, by, companion, lateinit, init,
    public, private, protected, internal, enum, open, abstract, sealed, data
  },
  sensitive=true,
  morecomment=[l]{//},
  morecomment=[s]{/*}{*/},
  morestring=[b]",
  morestring=[b]',
  morestring=[s]{"""*}{*"""},
}

\lstset{
  language=kotlin,
  basicstyle=\ttfamily\small,
  keywordstyle=\color{blue}\bfseries,
  commentstyle=\color{gray}\itshape,
  stringstyle=\color{orange},
  showstringspaces=false,
  breaklines=true,
  frame=single
}

%%% Вставляем по очереди все содержательные части документа %%%

\begin{document}

\thispagestyle{empty}

\begin{center}
    МИНИСТЕРСТВО ЦИФРОВОГО РАЗВИТИЯ, СВЯЗИ И МАССОВЫХ КОММУНИКАЦИЙ \\ РОССИЙСКОЙ ФЕДЕРАЦИИ

    \vspace{20pt}

    Федеральное государственное бюджетное образовательное учреждение  \\  высшего образования \\
    "<Сибирский государственный университет телекоммуникаций и информатики"> \\

    \vspace{20pt}

    Кафедра ИВТ \\  (ИА, ИП, ИВ)
\end{center}

\vfill

\begin{center}
    РЕФЕРАТ \\  
    по дисциплине \\
    \textit{"<Визуальное программирование">}

    \vspace{20pt}

    по теме: \\
    \uppercase{Gps Location}
\end{center}

\vfill

    \noindent Студент: \\
    \textit{Группа ИА331 \\ К.А. Любимов}

    \vspace{20pt}

    \noindent Предподаватель: \\
    \textit{\hfill Р.В. Ахпашев}

\vfill

\begin{center}
    Новосибирск 2025 г.
\end{center}                                     % Титульник

\newpage % Переходим на новую страницу
\setcounter{page}{2} % Начинаем считать номера страниц со второй
\OnehalfSpacing* % Задаем полуторный интервал текста (в титульнике одинарный, поэтому команда стоит после него)

\tableofcontents*                                   % Автособираемое оглавление

\chapter*{Введение}
\addcontentsline{toc}{chapter}{Введение}
\label{ch:intro}
    В рамках отчета рассматривается прилоежние, в котором реализовано \textbf{Gps Location}, разработанный на языке Kotlin с использованием Jetpack Compose для описания интерфейса.
    Программа представляет из себя простой и интуитивно понятный интерфейс на котором можно увидеть ширину и долготу устройства, которое использует приложение, а также приложение записывает полученные данные в формате JSON в память устройства
    \\
    Приложение демонстрирует работу со следующими компонентами: 
    \\
    \begin{itemize}
        \item \textbf{Jetpack Compose} — для построения UI с использованием современных декларативных подходов;
        \item \textbf{MediaStore API} — для доступа к фалам, хранящимся на устройстве;
        \item \textbf{Fused Location Provider API} — для получение геолокации;
        \item \textbf{Состояния Compose} — для вывода данных и обновления интерфейса.
    \end{itemize}
\endinput                                     % Введение
\chapter{Теория}
\label{ch:chap1}
В данной главе рассматриваются основные теоретические аспекты, лежащие в основе реализации музыкального плеера на языке программирования Kotlin с использованием современных подходов к разработке пользовательских интерфейсов.

\section{Jetpack Compose}

\textbf{Jetpack Compose} — современный инструмент Google для создания пользовательских интерфейсов на Android. В отличие от традиционного подхода с использованием XML, Jetpack Compose позволяет описывать UI декларативно с помощью кода на Kotlin. Это существенно упрощает реализацию интерактивных компонентов и их обновление при изменении данных.

Ключевые особенности Jetpack Compose:

\begin{itemize}
    \item \textbf{Декларативность}: UI автоматически перестраивается при изменении состояния.
    \item \textbf{Композиторы (Composables)}: функции с аннотацией \texttt{@Composable} определяют пользовательский интерфейс.
    \item \textbf{Состояние и реактивность}: при использовании \texttt{mutableStateOf} и других типов состояния интерфейс обновляется без ручного вмешательства.
    \item \textbf{Интеграция с ViewModel}: позволяет эффективно управлять состоянием приложения.
\end{itemize}

Jetpack Compose — мощный инструмент, заменяющий устаревшие методы построения UI, такой как XML и View Binding, и подходит для современных Android-приложений.

\section{Получение геолокации, и вывод}

LocationServices — это компонент Google Play Services, предоставляющий API для работы с геолокацией на Android. Он входит в состав библиотеки com.google.android.gms:play-services-location и предлагает более продвинутые и энергоэффективные методы получения данных о местоположении, чем стандартный LocationManager.

\begin{itemize}
    \item Проверка активности GPS/сети через \verb|LocationManager|
    \item Использование \verb|FusedLocationProviderClient|
    \item Получение последней известной локации (\verb|lastLocation|)
    \item Обновление UI через \verb|mutableStateOf|
\end{itemize}

\section{Работа с хранилищем файлов (MediaStore)}

Для получения списка файлов, хранящихся на устройстве, используется \textbf{MediaStore} — Android API, обеспечивающий доступ к общедоступным медиа-ресурсам.

\begin{itemize}
        \item Генерация JSON-файла с координатами и временем
        \item Запись в папку \verb|Downloads|:
            \begin{itemize}
                \item Для Android 10+: через \verb|MediaStore|
                \item Для старых версий: прямой доступ к \verb|File|
            \end{itemize}
        \item Обновление медиа-базы через \verb|MediaScannerConnection|
    \end{itemize}

Таким образом, обеспечивается сохранение JSON файлов в памяти устройства.

\section{Принципы взаимодействия компонентов}

Приложение запрашивает доступ к геолокации, проверяет активность GPS/сети, получает последние координаты через Google Play Services, сохраняет их в файл (с учётом версии Android) и позволяет вернуться на главный экран. Ошибки и статусы отображаются через уведомления.

Общий цикл работы:

\begin{enumerate}
    \item Старт активности → Запрос разрешения на геолокацию.
    \item Если разрешение есть → Проверка включённого GPS/сети.
    \item Если выключено → Уведомление пользователя.
    \item Если включено → Запрос последней локации.
    \item Успешный запрос → Сохранение координат в файл + обновление UI.
    \item Ошибка → Toast с описанием проблемы.
    \item Возврат на главный экран → По нажатию кнопки через Intent.
\end{enumerate}

Ключевые особенности:

\begin{enumerate}
    \item Асинхронная обработка данных.
    \item Минимальное потребление ресурсов (использование последней известной локации).
    \item Реактивное обновление интерфейса (Jetpack Compose).
    \item Адаптация под разные версии Android.
\end{enumerate}                                     % Первая глава
\chapter{Практика}
\label{ch:chap2}

В данной главе рассматриваются ключевые аспекты реализации GPS Loacation на языке Kotlin с использованием Jetpack Compose. 
Основное листингу проекта.

\section{Среда разработки}

Для реализации проекта использовалась следующая среда:

\begin{itemize}
    \item \textbf{Язык программирования}: Kotlin
    \item \textbf{Среда разработки (IDE)}: Android Studio
    \item \textbf{Минимальная версия SDK}: 21 (Android 5.0 Lollipop)
    \item \textbf{Целевая версия SDK}: 33 (Android 13)
    \item \textbf{Библиотеки и технологии:}
    \begin{itemize}
        \item Jetpack Compose (UI)
        \item FusedLocationProviderClient (Google Play Services)
        \item LocationManager (проверка статуса GPS/сети)
        \item MediaStore и File API (сохранение данных)
    \end{itemize}
\end{itemize}

\section{MapPage.kt}

Приложение состоит из следующих ключевых компонентов:

\begin{itemize}
    \item \textbf{MapPage.kt} — Основная активность, в которой инициализируется Compose-интерфейс.
    \item \textbf{LocationScreen() (Compose-функция)} — Реализует UI: отображение координат, кнопка возврата.
    \item \textbf{Взаимодействие с локацией:} — FusedLocationProviderClient для получения последней известной локации. LocationManager для проверки активности GPS/сети.
\end{itemize}

Состояния и данные:

\begin{itemize}
    \item \textbf{Текущие координаты} — широта и долгота (обновляются через mutableStateOf).
    \item \textbf{Статус разрешений} — \texttt{ACCESS\_FINE\_LOCATION} (обрабатывается через \texttt{ActivityResultLauncher}).
    \item \textbf{Статус геолокации} — проверка включения GPS/сети.
\end{itemize}

\begin{lstlisting}[language=Kotlin, caption=MapPage.kt]

package com.example.mycal.activities

import android.Manifest
import android.content.ContentValues
import android.content.Context
import android.content.Intent
import android.content.pm.PackageManager
import android.location.LocationManager
import android.os.Bundle
import android.os.Environment
import android.provider.MediaStore
import android.widget.Toast
import androidx.activity.ComponentActivity
import androidx.activity.compose.setContent
import androidx.activity.compose.rememberLauncherForActivityResult
import androidx.activity.enableEdgeToEdge
import androidx.activity.result.contract.ActivityResultContracts
import androidx.compose.foundation.layout.*
import androidx.compose.material3.Button
import androidx.compose.material3.ButtonDefaults
import androidx.compose.material3.ExperimentalMaterial3Api
import androidx.compose.material3.MaterialTheme
import androidx.compose.material3.Surface
import androidx.compose.material3.Text
import androidx.compose.runtime.*
import androidx.compose.ui.Alignment
import androidx.compose.ui.Modifier
import androidx.compose.ui.graphics.Color
import androidx.compose.ui.platform.LocalContext
import androidx.compose.ui.unit.dp
import androidx.core.content.ContextCompat
import com.example.mycal.MainPage
import com.example.mycal.ui.theme.MycalTheme
import com.example.mycal.ui.theme.Rose
import com.example.mycal.ui.theme.Russian_Violete
import com.google.android.gms.location.LocationServices
import org.json.JSONObject

class LocationActivity : ComponentActivity() {

    @OptIn(ExperimentalMaterial3Api::class)
    override fun onCreate(savedInstanceState: Bundle?) {
        super.onCreate(savedInstanceState)
        enableEdgeToEdge()

        setContent {
            MycalTheme {
                Surface(modifier = Modifier.fillMaxSize(), color = Russian_Violete) {
                    LocationScreen()
                }
            }
        }
    }

    @Composable
    fun LocationScreen() {
        val context = LocalContext.current

        var latText by remember { mutableStateOf("—") }
        var lonText by remember { mutableStateOf("—") }
        var hasPermission by remember { mutableStateOf(false) }

        val permissionLauncher = rememberLauncherForActivityResult(
            contract = ActivityResultContracts.RequestPermission()
        ) { granted ->
            hasPermission = granted
            if (!granted) {
                Toast.makeText(
                    context,
                    "Без разрешения получение локации невозможно",
                    Toast.LENGTH_SHORT
                ).show()
            }
        }

        LaunchedEffect(Unit) {
            permissionLauncher.launch(Manifest.permission.ACCESS_FINE_LOCATION)
        }

        LaunchedEffect(hasPermission) {
            if (hasPermission) {
                val lm = context.getSystemService(Context.LOCATION_SERVICE) as LocationManager
                val enabled = lm.isProviderEnabled(LocationManager.GPS_PROVIDER)
                        || lm.isProviderEnabled(LocationManager.NETWORK_PROVIDER)
                if (!enabled) {
                    Toast.makeText(
                        context,
                        "Включите геолокацию в настройках, пожалуйста",
                        Toast.LENGTH_LONG
                    ).show()
                } else {
                    val permissionGranted = ContextCompat.checkSelfPermission(
                        context,
                        Manifest.permission.ACCESS_FINE_LOCATION
                    ) == PackageManager.PERMISSION_GRANTED

                    if (permissionGranted) {
                        val client = LocationServices.getFusedLocationProviderClient(context)
                        client.lastLocation
                            .addOnSuccessListener { location ->
                                if (location != null) {
                                    latText = location.latitude.toString()
                                    lonText = location.longitude.toString()
                                    saveCoordsToDownloads(context, location.latitude, location.longitude)
                                } else {
                                    Toast.makeText(
                                        context,
                                        "Не удалось получить локацию",
                                        Toast.LENGTH_SHORT
                                    ).show()
                                }
                            }
                            .addOnFailureListener {
                                Toast.makeText(
                                    context,
                                    "Ошибка при получении локации",
                                    Toast.LENGTH_SHORT
                                ).show()
                            }
                    } else {
                        Toast.makeText(
                            context,
                            "Разрешение на геолокацию не получено",
                            Toast.LENGTH_SHORT
                        ).show()
                    }
                }
            }
        }

        Column(
            modifier = Modifier
                .fillMaxSize()
                .padding(24.dp),
            verticalArrangement = Arrangement.SpaceBetween,
            horizontalAlignment = Alignment.CenterHorizontally
        ) {
            Column(
                horizontalAlignment = Alignment.CenterHorizontally
            ) {
                Text(
                    "Текущие координаты",
                    style = MaterialTheme.typography.headlineMedium,
                    color = Color.White
                )
                Spacer(Modifier.height(16.dp))
                Text("Широта: $latText", style = MaterialTheme.typography.bodyLarge, color = Color.White)
                Text("Долгота: $lonText", style = MaterialTheme.typography.bodyLarge, color = Color.White)
            }

            Button(
                onClick = {
                    context.startActivity(Intent(context, MainPage::class.java))
                },
                colors = ButtonDefaults.buttonColors(Rose),
                modifier = Modifier
                    .fillMaxWidth()
                    .height(56.dp)
            ) {
                Text("Back to Main", color = Color.White)
            }
        }
    }

    private fun saveCoordsToDownloads(context: Context, lat: Double, lon: Double) {
        val json = JSONObject().apply {
            put("latitude", lat)
            put("longitude", lon)
            put("timestamp", System.currentTimeMillis())
        }
        val filename = "coords_${System.currentTimeMillis()}.json"
        val bytes = json.toString(2).toByteArray()

        val values = ContentValues().apply {
            put(MediaStore.Downloads.DISPLAY_NAME, filename)
            put(MediaStore.Downloads.MIME_TYPE, "application/json")
            put(MediaStore.Downloads.RELATIVE_PATH, Environment.DIRECTORY_DOWNLOADS)
        }
        val uri = context.contentResolver.insert(
            MediaStore.Downloads.EXTERNAL_CONTENT_URI, values
        )
        if (uri != null) {
            context.contentResolver.openOutputStream(uri).use { it?.write(bytes) }
            Toast.makeText(
                context,
                "Координаты сохранены в Downloads/$filename",
                Toast.LENGTH_LONG
            ).show()
        } else {
            Toast.makeText(context, "Не удалось создать файл в Downloads", Toast.LENGTH_SHORT).show()
        }
    }
}

\end{lstlisting}

Интерфейс реализован с помощью Jetpack Compose. В нём отображаются:

\begin{itemize}
    \item \textbf{Основной контент}
        \begin{itemize}
            \item Заголовок "Текущие координаты" (белый цвет, крупный шрифт)
            \item Поле "Широта: —" (белый цвет)
            \item Поле "Долгота: —" (белый цвет)
        \end{itemize}
    \item \textbf{Кнопка внизу экрана}
        \begin{itemize}
            \item Текст "Back to Main" (белый цвет)
            \item Розовый фон (цвет Rose)
            \item Занимает всю ширину экрана
            \item Высота 56dp
        \end{itemize}
\end{itemize}

Общий принцип работы:

\begin{itemize}
    \item  \textbf{Инициализация и запрос разрешений:}
        \begin{itemize}
            \item При старте активити автоматически запускается запрос разрешения \texttt{ACCESS\_FINE\_LOCATION}
            \item Используется ActivityResultLauncher для обработки ответа пользователя
            \item Состояние hasPermission отслеживает наличие/отсутствие разрешения
        \end{itemize}
    \item \textbf{Проверка доступности геолокации:}
        \begin{itemize}
            \item После получения разрешения проверяется статус GPS/сетевых провайдеров
            \item Если геолокация отключена - показывается Toast с просьбой включить
            \item При успешной проверке переходит к получению координат
        \end{itemize}
    \item \textbf{Получение местоположения:}
        \begin{itemize}
            \item Используется FusedLocationProviderClient для доступа к API локации
            \item Запрашивается последнее известное местоположение (lastLocation)
            \item Реализованы коллбеки:
            \begin{itemize}
                \item onSuccessListener - обработка успешного получения координат
                \item onFailureListener - обработка ошибок получения
            \end{itemize}
        \end{itemize}
    \item \textbf{Обновление UI и сохранение данных:}
        \begin{itemize}
            \item Полученные координаты обновляют состояние latText/lonText
            \item Вызывается saveCoordsToDownloads() для сохранения в файл:
            \begin{itemize}
                \item Создается JSON-объект с координатами и временной меткой
                \item Используется MediaStore API для сохранения в папку Downloads
                \item Имя файла генерируется по шаблону \texttt{coords\_<timestamp>.json}
            \end{itemize}
        \end{itemize}
    \item \textbf{Обратная связь с пользователем:}
        \begin{itemize}
            \item Toast-уведомления о статусе операций
            \item Визуальное отображение координат в реальном времени
            \item Обработка ошибок (отсутствие разрешения, проблемы с записью файла)
        \end{itemize}
    \item \textbf{Навигация:}
        \begin{itemize}
            \item Кнопка "Back to Main" запускает MainPage через явный Intent
            \item Сохранение данных происходит асинхронно, не блокируя UI
        \end{itemize}
    \item \textbf{Ключевые технологии:}
        \begin{itemize}
            \item Jetpack Compose для UI
            \item Location Services API (Fused Location Provider)
            \item Система разрешений Android (Runtime Permissions)
            \item ContentResolver для работы с MediaStore
            \item JSON для сериализации данных
            \item Асинхронная обработка через Listeners
        \end{itemize}
    \item \textbf{Поток данных:}
        \begin{itemize}
            \item Запрос разрешения → Проверка GPS → Получение локации → Обновление UI → Сохранение в файл → Возможность возврата на главный экран
        \end{itemize}
\end{itemize}

Особенность: Все операции с геолокацией и файловой системой выполняются асинхронно, что предотвращает блокировку основного потока UI.
                                     % Вторая глава
\chapter*{Заключение}
\addcontentsline{toc}{chapter}{Заключение}

В результате выполнения работы было разработано Android-приложение для получения и сохранения геолокационных данных с использованием современных инструментов и технологий. Я опирался на репозиторий преподавателя \cite{GitHub}, на видеоролики из ютуба \cite{YouTube} и сайт Metonit \cite{Metanit}. В качестве фреймворка для пользовательского интерфейса использован \textbf{Jetpack Compose}, что позволило создать гибкий и адаптивный интерфейс с декларативным подходом.

Основной функционал приложения включает:
\begin{itemize}
    \item Автоматический запрос разрешений на доступ к геолокации через \textbf{Runtime Permissions}.
    \item Отображение текущих координат (широты и долготы) в реальном времени.
    \item Сохранение данных в формате \textbf{JSON} в системную папку \texttt{Downloads} через \textbf{MediaStore API}.
    \item Интеграцию с \textbf{FusedLocationProviderClient} для точного определения местоположения.
\end{itemize}

При разработке были применены лучшие практики проектирования:
\begin{itemize}
    \item Асинхронная обработка операций с использованием коллбеков и \textbf{Listeners}.
    \item Реализация устойчивости к изменениям конфигурации устройства.
    \item Организация кода с учётом принципов расширяемости и поддерживаемости.
\end{itemize}

Полученный результат демонстрирует корректную работу всех компонентов: успешное получение координат, их визуализацию в интерфейсе, сохранение в файл с уникальным именем, а также обработку ошибок при отсутствии разрешений или отключённой геолокации. Пользовательский интерфейс обеспечивает интуитивное взаимодействие, а система уведомлений через \textbf{Toast} информирует о статусе операций.

Разработка позволила получить практические навыки:
\begin{itemize}
    \item Работы с геолокацией в Android, включая взаимодействие с \textbf{Google Play Services}.
    \item Создания адаптивных интерфейсов на базе \textbf{Jetpack Compose}.
    \item Управления файловой системой через \textbf{ContentResolver} и \textbf{MediaStore}.
\end{itemize}

Перспективы развития приложения включают интеграцию с картографическими сервисами (например, \textbf{Google Maps}), реализацию фонового отслеживания локации, а также добавление синхронизации данных с облачными хранилищами. Данная работа подтверждает возможность создания устойчивых и масштабируемых решений в области мобильной разработки на платформе Android.                                     % Третья глава

\printbibliography[title=Список использованных источников] % Автособираемый список литературы

\end{document}
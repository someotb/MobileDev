\chapter{Теория}
\label{ch:chap1}
В данной главе рассматриваются основные теоретические аспекты, лежащие в основе реализации музыкального плеера на языке программирования Kotlin с использованием современных подходов к разработке пользовательских интерфейсов.

\section{Jetpack Compose}

\textbf{Jetpack Compose} — современный инструмент Google для создания пользовательских интерфейсов на Android. В отличие от традиционного подхода с использованием XML, Jetpack Compose позволяет описывать UI декларативно с помощью кода на Kotlin. Это существенно упрощает реализацию интерактивных компонентов и их обновление при изменении данных.

Ключевые особенности Jetpack Compose:

\begin{itemize}
    \item \textbf{Декларативность}: UI автоматически перестраивается при изменении состояния.
    \item \textbf{Композиторы (Composables)}: функции с аннотацией \texttt{@Composable} определяют пользовательский интерфейс.
    \item \textbf{Состояние и реактивность}: при использовании \texttt{mutableStateOf} и других типов состояния интерфейс обновляется без ручного вмешательства.
    \item \textbf{Интеграция с ViewModel}: позволяет эффективно управлять состоянием приложения.
\end{itemize}

Jetpack Compose — мощный инструмент, заменяющий устаревшие методы построения UI, такой как XML и View Binding, и подходит для современных Android-приложений.

\section{Получение геолокации, и вывод}

LocationServices — это компонент Google Play Services, предоставляющий API для работы с геолокацией на Android. Он входит в состав библиотеки com.google.android.gms:play-services-location и предлагает более продвинутые и энергоэффективные методы получения данных о местоположении, чем стандартный LocationManager.

\begin{itemize}
    \item Проверка активности GPS/сети через \verb|LocationManager|
    \item Использование \verb|FusedLocationProviderClient|
    \item Получение последней известной локации (\verb|lastLocation|)
    \item Обновление UI через \verb|mutableStateOf|
\end{itemize}

\section{Работа с хранилищем файлов (MediaStore)}

Для получения списка файлов, хранящихся на устройстве, используется \textbf{MediaStore} — Android API, обеспечивающий доступ к общедоступным медиа-ресурсам.

\begin{itemize}
        \item Генерация JSON-файла с координатами и временем
        \item Запись в папку \verb|Downloads|:
            \begin{itemize}
                \item Для Android 10+: через \verb|MediaStore|
                \item Для старых версий: прямой доступ к \verb|File|
            \end{itemize}
        \item Обновление медиа-базы через \verb|MediaScannerConnection|
    \end{itemize}

Таким образом, обеспечивается сохранение JSON файлов в памяти устройства.

\section{Принципы взаимодействия компонентов}

Приложение запрашивает доступ к геолокации, проверяет активность GPS/сети, получает последние координаты через Google Play Services, сохраняет их в файл (с учётом версии Android) и позволяет вернуться на главный экран. Ошибки и статусы отображаются через уведомления.

Общий цикл работы:

\begin{enumerate}
    \item Старт активности → Запрос разрешения на геолокацию.
    \item Если разрешение есть → Проверка включённого GPS/сети.
    \item Если выключено → Уведомление пользователя.
    \item Если включено → Запрос последней локации.
    \item Успешный запрос → Сохранение координат в файл + обновление UI.
    \item Ошибка → Toast с описанием проблемы.
    \item Возврат на главный экран → По нажатию кнопки через Intent.
\end{enumerate}

Ключевые особенности:

\begin{enumerate}
    \item Асинхронная обработка данных.
    \item Минимальное потребление ресурсов (использование последней известной локации).
    \item Реактивное обновление интерфейса (Jetpack Compose).
    \item Адаптация под разные версии Android.
\end{enumerate}
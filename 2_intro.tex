\chapter*{Введение}
\addcontentsline{toc}{chapter}{Введение}
\label{ch:intro}
    В рамках отчета рассматривается прилоежние, в котором реализовано \textbf{Gps Location}, разработанный на языке Kotlin с использованием Jetpack Compose для описания интерфейса.
    Программа представляет из себя простой и интуитивно понятный интерфейс на котором можно увидеть ширину и долготу устройства, которое использует приложение, а также приложение записывает полученные данные в формате JSON в память устройства
    \\
    Приложение демонстрирует работу со следующими компонентами: 
    \\
    \begin{itemize}
        \item \textbf{Jetpack Compose} — для построения UI с использованием современных декларативных подходов;
        \item \textbf{MediaStore API} — для доступа к фалам, хранящимся на устройстве;
        \item \textbf{Fused Location Provider API} — для получение геолокации;
        \item \textbf{Состояния Compose} — для вывода данных и обновления интерфейса.
    \end{itemize}
\endinput